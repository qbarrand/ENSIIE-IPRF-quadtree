\documentclass[11pt]{scrartcl}

% French
\usepackage[utf8]{inputenc}
\usepackage[T1]{fontenc}
\usepackage[frenchb]{babel}

\usepackage{url}
\usepackage{hyperref}
\hypersetup{
    colorlinks,
    linkcolor={red!50!black},
    citecolor={blue!50!black},
    urlcolor={blue!30!black}
}
\usepackage{hyphenat}

\usepackage{dirtree}
\usepackage{enumitem}

\usepackage[usenames,dvipsnames]{xcolor}

\usepackage{fancyhdr}
\pagestyle{fancy}
\fancyhf{}
\fancyhead[LE,RO]{Quentin Barrand}
\fancyhead[RE,LO]{IPRF - « QuadTree »}
\fancyfoot[LE,CO]{\thepage}

% Specifying the outputdir option else a bug prevents from compiling a document that contains in-doc minted code.
\usepackage[outputdir=/tmp]{minted}

\newcommand{\functionname}[1]{\texttt{\textcolor{MidnightBlue}{#1}}}
\newcommand{\filename}[1]{\texttt{\textcolor{RawSienna}{#1}}}
\newcommand{\code}[1]{\mintinline{ocaml}{#1}}

%% This code substitutes underscore.sty
\begingroup\lccode`~=`_
\lowercase{\endgroup
  \protected\def~{\ifmmode\sb\else\textunderscore\fi}
}
\AtBeginDocument{\catcode`_=\active}
%% end

\begin{document}

\title{\textbf{IPRF - « QuadTree »}}
\subtitle{Rapport}

\author{Quentin Barrand\\
		\href{mailto:quentin@quba.fr}{\texttt{quentin@quba.fr}}}
		
\date{\today}

\maketitle

\begin{abstract}
Ce document a été réalisé dans le cadre du projet final du module IPRF \mbox{« Programmation fonctionnelle »} de la formation FIPA de l'ENSIIE.\\
Dans ce projet, nous étudions l'opportunité d'utiliser une structure de données de type arbre 4-aire (« QuadTree ») pour stocker des points. Nous voyons
\end{abstract}

\break

\section*{Organisation du projet}
Pour chaque section, j'ai décidé de placer les fonctions à écrire dans les fichiers \filename{partX.ml} et le code interactif (tel que les tests graphiques) dans les fichiers \filename{partX.test.ml}. Cette organisation permet d'importer le code des sections précédentes à l'aide de directives \mintinline{ocaml}{#use} sans provoquer d'appels de fonction innattendus.
\\
Le rendu est donc constitué des fichiers suivants :
\dirtree{%
 .1 /.
 .2 coord.ml\DTcomment{Partie 1}.
 .2 display.ml\DTcomment{Fourni par le sujet}.
 .2 part3.ml\DTcomment{Partie 3}.
 .2 part3.test.ml\DTcomment{Partie 3 - tests}.
 .2 part4.ml\DTcomment{Partie 4}.
 .2 part4.test.ml\DTcomment{Partie 4 - tests}.
 .2 part5.ml\DTcomment{Partie 5}.
 .2 quadtree.ml\DTcomment{Partie 2}.
 .2 rect.ml\DTcomment{Partie 1}.
 .2 report.tex\DTcomment{Le présent rapport}.
 .2 simulation1.ml\DTcomment{Fourni par le sujet}.
 .2 simulation1.ml\DTcomment{Fourni par le sujet}.
}


\section{Échauffement sur les rectangles}

\begin{description}
\item[Question 1.] Voir \filename{rect.ml}.

\item[Question 2.] Voir \filename{rect.ml}.

\item[Question 3.] Voir \filename{rect.ml}.

\item[Question 4.] Voir \filename{rect.ml}.

\item[Question 5.] Voir \filename{rect.ml}.

\item[Question 6.] Voir \filename{rect.ml}.
\end{description}

\section{La structure de données QuadTree}

\begin{description}
\item[Question 7.] Pour stocker une collection de points, on pourrait utiliser quatre grandes familles de structures de données :
\begin{description}
\item[Ensembles ou listes] On stocke les coordonnées des points les uns à la suite des autres dans une liste.\\
Occupation mémoire : faible ($n \times taille(n)$).\\
Algorithmes d'accès : très peu efficaces (parcours de la liste - $\mathcal{O}(n)$).
\item[Dictionnaires] On stocke les points dans un dictionnaire dont les clés sont les coordonnées des points.\\
Occupation mémoire : Modérée (on considère que pour éviter la plupart des collisions sur la fonction de hachage, il est nécessaire de d'utiliser un ensemble environ 5 fois plus grand que le nombre d'éléments à stocker -- $5 \times n \times taille(n)$).\\
Algorithmes d'accès : très efficace lorsque l'on connaît les coordonnées exactes du point (temps quasi-constant), peu efficaces dans les autres cas (recherche de tous les points contenus dans une restriction de l'espace initial par exemple) puisqu'il est nécessaire de parcourir tout le dictionnaire ($\mathcal{O}(n)$).
\item[Matrices] On stocke un tableau à deux dimensions en mémoire, et aux coordonnées du point dans la matrice on stocke l'objet lié au point. Toutefois, cette structure de données peut se révéler inadaptée aux stockage de points aux coordonnées représentées par des nombres flottants, en raison de la façon dont ils sont représentés en mémoire.\\
Occupation mémoire : très importante (de l'ordre de $((Abs_{max} - Abs_{min}) \times (Ord_{max} - Ord_{min})) \times taille(n)$).\\
Algorithmes d'accès : très efficaces (temps quasi-constant qui dépend seulement du nombre d'éléments possédant la même clé).
\item[Arbres n-aires] Les arbres permettent de combiner une utilisation mémoire modérée et une recherche efficace. L'emploi du QuadTree, arbre 4-aire, est particulèrement justifié pour travailler avec des points 2D, puisqu'il découpe si nécessaire l'espace en rectangles successifs et que tout point est logiquement contenu dans un rectangle. Ainsi, la recherche de tous les points contenus dans une restriction de l'espace initial est très rapide.\\
Occupation mémoire : modérée ($\leq 4 \times n \times taille(n)$).\\
Algorithmes d'accès : efficaces ($\mathcal{O}(\log{}n)$).
\end{description}
\item[Question 8.] Voir \filename{quadtree.ml}.

\item[Question 9.] Voir \filename{quadtree.ml}.

\item[Question 10.] Voir \filename{quadtree.ml}.

\item[Question 11.] Voir \filename{quadtree.ml}.

\item[Question 12.] Voir \filename{quadtree.ml}.

\item[Question 13.] Voir \filename{quadtree.ml}.\\
Intuitivement, on pourrait être tenté d'utiliser \functionname{list_of_quadtree}, puis de supprimer l'élément choisi dans la liste, puis de faire de nouveau appel à \functionname{quadtree_of_list}; la fonction \functionname{list_remove} implémente cette méthode.\\
Pour éviter l'opération en $\mathcal{O}(\log{}n)$ qui consiste à reconstruire le carré après avoir supprimé un objet dans la liste, j'ai choisi d'écrire une fonction \functionname{clean_qt} qui devrait permettre d'avoir une simplification plus performante du QuadTree que si l'on utilisait la méthode décrite plus haut. Malgré de nombreux tests, je n'ai pas trouvé de cas dans lesquels cette fonction produirait des résultats inattendus.
\end{description}

\section{Représentation graphique d'un QuadTree et tests}

\begin{description}
\item[Question 14.] Fonctions de \filename{display.ml} :
\begin{description}
\item[\functionname{draw_data}] Dessine une donnée stockée dans une feuille \code{Leaf of coord * 'a} d'un QuadTree. La fonction de conversion de \code{'a} vers \code{String} ainsi que la couleur de dessin doivent être passées en paramètre.
\item[\functionname{draw_quadtree}] Dessine récursivement un QuadTree. Commence par dessiner en bleu le rectangle dans lequel est contenu le QuadTree, puis, en fonction de la nature de l'élement \code{cell} :
\begin{itemize}
\item ne fait rien si l'élément est \code{Empty};
\item fait appel à \functionname{draw_data} si l'élément est \code{Leaf};
\item s'appelle récursivement pour chaque sous-QuadTree si l'élément est \code{Node}.
\end{itemize}
\item[\functionname{wait_and_quit}] Attend un appui sur une touche du clavier puis ferme la fenêtre de dessin.
\item[\functionname{init}] Initialise un environnement de dessin en fonction de la taille du rectangle passé en paramètre et ouvre la fenêtre. Retourne un triplet de \code{Float} nommé \code{dparams} dans le reste du projet et utilisé par toutes les fonctions de dessin. 
\item[\functionname{simple_test}] Appelle la fonction \functionname{draw_quadtree} à l'aide du triplet \code{dparams} (obtenu par un précédent appel à la fonction \functionname{init}) et d'une fonction de conversion vers le type \code{String} pour dessiner le QuadTree passé en paramètre.
\end{description}

\item[Question 15.] Voir \filename{part3.ml}.

\item[Question 16.] TODO
\end{description}

\section{Placement du disque}

\begin{description}
\item[Question 17.] Voir \filename{part4.ml}.

\item[Question 18.] Voir \filename{part4.ml}.

\item[Question 19.] Voir \filename{part4.ml}.

\item[Question 20.] Fonctions de \filename{simulation1.ml} :
\begin{description}
\item[\functionname{get_point}] Retourne les coordonnées du point sur lequel se trouve le pointeur de la souris lors du déclenchement du clic.
\item[\functionname{get_disk}] Retourne les coordonnées du centre et la longueur du rayon du cercle désigné par une action « clic - glisser » (début du clic - déplacement du pointeur - relâchement du pointeur).
\item[\functionname{draw_disk}] Dessine à disque d'une couleur spécifiée à des coordonnées spécifiées, plein ou non.
\item[\functionname{draw_disk_with_collisions}] Dessine un disque :
\begin{itemize}
\item de couleur jaune s'il recouvre des points du QuadTree (obtenus grâce à \functionname{collision_disk} de \filename{part4.ml}), et fait appel à \functionname{draw_data} de \filename{display.ml} pour dessiner ces points en rouge le cas échéant;
\item de couleur verte sinon.
\end{itemize}
\item[\functionname{simulation_placement}] Fait appel aux fonctions \functionname{init} et \functionname{draw_quadtree} de \functionname{display.ml} et aux fonctions \functionname{get_disk} et \functionname{draw_disk_with_collisions} pour dessiner un cercle défini à la souris par l'utilisateur et ses éventuels collisions avec les points du QuadTree.
\end{description}
\end{description}

\section{Déplacement du disque et détection de collision}

\begin{description}
\item[Question 21.] Voir \filename{part5.ml}.

\item[Question 22.] Voir \filename{part5.ml}.

\item[Question 23.] Fonctions de \filename{simulation2.ml} :
\begin{description}
\item[\functionname{draw_trail_with_collisions}]
\item[\functionname{simulation_move}] Idem que \functionname{simulation_placement} de \filename{simulation1.ml}, sauf qu'après avoir défini un cercle, un second clic de souris utilise la fonction \functionname{draw_trail_with_collisions} pour déplacer le cercle vers cette seconde position et afficher en rouge les points survolés pendant son déplacement.
\end{description}
\end{description}

\break

\section*{Bonus}

\end{document}
