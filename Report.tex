\documentclass[11pt]{scrartcl}

% French
\usepackage[utf8]{inputenc}
\usepackage[T1]{fontenc}
\usepackage[frenchb]{babel}

\usepackage{xcolor}
\usepackage{url}
\usepackage{hyperref}
\hypersetup{
    colorlinks,
    linkcolor={red!50!black},
    citecolor={blue!50!black},
    urlcolor={blue!30!black}
}
\usepackage{hyphenat}

\usepackage{enumitem}

\usepackage{fancyhdr}
\pagestyle{fancy}
\fancyhf{}
\fancyhead[LE,RO]{Quentin Barrand}
\fancyhead[RE,LO]{IPRF - « QuadTree »}
\fancyfoot[LE,CO]{\thepage}

% Specifying the outputdir option else a bug prevents from compiling a document that contains in-doc minted code.
\usepackage[outputdir=/tmp]{minted}

\title{\textbf{IPRF - « QuadTree »}}
\subtitle{Rapport}


\author{Quentin Barrand\\
		\href{mailto:quentin@quba.fr}{\texttt{quentin@quba.fr}}}
		
\date{\today}

\begin{document}

\maketitle

\begin{abstract}
Résumé
\end{abstract}

\break

\section*{Organisation du projet}

\section{Échauffement sur les rectangles}

\begin{description}
\item[Question 1.]
\item[Question 2.]
\item[Question 3.]
\item[Question 4.]
\item[Question 5.]
\item[Question 6.]
\end{description}

\section{La structure de données QuadTree}

\begin{description}
\item[Question 7.] \hfill \\
	\begin{description}
	\item[Avantages.]
	\item[Inconvénients.]
	\end{description}
\item[Question 8.]
\item[Question 9.]
\item[Question 10.]
\item[Question 11.]
\item[Question 12.]
\item[Question 13.]
\end{description}

\section{Représentation graphique d'un QuadTree et tests}

\begin{description}
\item[Question 14.]
\item[Question 15.]
\item[Question 16.]
\end{description}

\section{Placement du disque}

\begin{description}
\item[Question 17.]
\item[Question 18.]
\item[Question 19.]
\item[Question 20.]
\end{description}

\section{Déplacement du disque et détection de collision}

\begin{description}
\item[Question 21.]
\item[Question 22.]
\item[Question 23.]
\end{description}
\end{document}
